\begin{SCn}
	
\scnheader{Агент установки спецификаций многократно используемых компонентов}
\scnidtf{sc-агент установки спецификаций многократно используемых компонентов}
\scnrelfrom{команда, вызывающая агент}{components init}
\begin{scnrelfromvector}{задача}
	\scnfileitem{Отыскать все спецификаци спецификаций многократно используемых компонентов в базе знаний.}
	\scnfileitem{Скачать спецификаций на устройство.}
\end{scnrelfromvector}	
\scnrelfrom{пример входной конструкции}{\scnfileimage[20em]{images/ci.png}}
\begin{scnrelfromset}{аргументы агента}
	\scnitem{пустое множество}
\end{scnrelfromset}
\scnrelfrom{ответ агента}{ответ агента установки спецификаций многократно используемых компонентов}
\begin{scnindent}
	\scntext{примечание}{В результате выполнения агентом поискового действия на устройстве появятся scs-файлы со спецификациями, скаченными по сети. Данные файлы сразу транслируются в sc-память.}
\end{scnindent}
\scnrelfrom{пример выходной конструкции}{\scnfileimage[20em]{images/ci_out.png}}

\scnheader{Агент установки компонентов}
\scnidtf{sc-агент установки компонентов}
\scnrelfrom{команда, вызывающая агент}{components install}
\begin{scnrelfromvector}{задача}
	\scnfileitem{Поиск спецификации компонента в базе знаний по заданному критерию, если он есть, иначе поиск всех спецификаций компонентов.}
	\scnfileitem{Скачивание и установка (опционально) на устройство.}
\end{scnrelfromvector}
	\scnrelfrom{пример входной конструкции}{\scnfileimage[20em]{images/cinst.png}}
\begin{scnrelfromset}{аргументы агента}
	\scnitem{множество компонентов}
	\begin{scnindent}
		\scntext{примечание}{В этом случае отыскиваются все компоненты, принадлежащие указанному множеству, и скачиваются.}
	\end{scnindent}
	\scnitem{множество множеств компонентов}
	\begin{scnindent}
		\scntext{примечание}{В этом случае отыскиваются все компоненты, принадлежащие указанным множествам множества, и скачиваются.}
	\end{scnindent}
\end{scnrelfromset}
\scnrelfrom{ответ агента}{ответ агента установки компонентов}
\begin{scnindent}
	\scntext{примечание}{В результате выполнения агентом действия по установке компонентов на устройстве появятся компоненты в соответствующих папках со спецификациями, и данные компоненты будут установлены, если это необходимо.}
\end{scnindent}
\scnrelfrom{пример выходной конструкции}{\scnfileimage[20em]{images/cinst_out.png}}

\scnheader{Агент поиска спецификаций компонентов}
\scnidtf{sc-агент поиска спецификаций компонентов}
\scnrelfrom{команда, вызывающая агент}{components search}
\begin{scnrelfromvector}{задача}
	\scnfileitem{Отыскать спецификации компонентов в базе знаний по заданным критериям.}
\end{scnrelfromvector}
\scnrelfrom{пример входной конструкции}{\scnfileimage[20em]{images/cs.png}}
\begin{scnrelfromset}{аргументы агента}
	\scnitem{authors}
	\begin{scnindent}
		\scntext{примечание}{В этом случае отыскивается спецификация компонента по автору, который создал компонент.}
	\end{scnindent}
		\scnitem{classes}
	\begin{scnindent}
		\scntext{примечание}{В этом случае отыскивается спецификация компонента по классу, которому принадлежит компонент.}
	\end{scnindent}
		\scnitem{explanations}
	\begin{scnindent}
		\scntext{примечание}{В этом случае отыскивается спецификация компонента по его описанию (примечанию), которое в общем случае может являться подстрокой реального примечания компонента, хранящегося в базе знаний.}
	\end{scnindent}
	\scnitem{пустое множество}
	\begin{scnindent}
		\scntext{примечание}{В этом случае ничего не будет найдено.}
	\end{scnindent}
\end{scnrelfromset}
\scnrelfrom{ответ агента}{ответ агента поиска компонентов}
\begin{scnindent}
	\scntext{примечание}{В результате выполнения агентом поискового действия сформируется ответ со множеством найденных компонентов.}
\end{scnindent}
\scnrelfrom{пример выходной конструкции}{\scnfileimage[20em]{images/cs_out.png}}

\scnheader{Агент поиска спецификации компонента по компоненту}
\scnidtf{sc-агент поиска спецификации компонента по компоненту}
\begin{scnrelfromvector}{задача}
	\scnfileitem{Отыскать спецификацию компонента в базе знаний по заданному компоненту.}
\end{scnrelfromvector}
\scnrelfrom{пример входной конструкции}{\scnfileimage[20em]{images/ss.png}}
\begin{scnrelfromvector}{аргументы агента}
	\scnitem{component}
	\begin{scnindent}
		\scntext{примечание}{Узел компонента, для которого необходимо найти спецификацию.}
	\end{scnindent}
\end{scnrelfromvector}
\scnrelfrom{ответ агента}{ответ агента поиска спецификации компонента по компоненту}
\begin{scnindent}
	\scntext{примечание}{В результате выполнения агентом поискового действия сформируется ответ с найденной спецификацией.}
\end{scnindent}
\scnrelfrom{пример выходной конструкции}{\scnfileimage[20em]{images/ss_out.png}}
\end{SCn}